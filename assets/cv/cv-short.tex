%%%%%%%%%%%%%%%%% PREAMBLE %%%%%%%%%%%%%%%%%%%%%%%%%%%%
%Change the font size of your document - 10pt, 12.1pt, etc.
\documentclass[letterpaper,11pt,oneside]{article}
\usepackage[utf8]{inputenc}
\usepackage{setspace}
\usepackage{hyperref}
\hypersetup{
  hidelinks
}
\usepackage{longtable}

\usepackage{graphicx}
\graphicspath{ {images/}} %upload your signature to this file
%Change the margins to fit your CV/resume content
\usepackage[left=1in, right=1in, bottom=0.7in, top=0.7in]{geometry}

%Skype information - include your Skype name for a link to add you on Skype
\newcommand*{\Skype}{\href{skype:john.smith?add}{john.smith}}
\newcommand{\Absender}[1][\normalsize]{\Skype}

%Changes the page numbers - {arabic}=arabic numerals, {gobble}=no page numbers, {roman}=Roman numerals
\pagenumbering{gobble}

%%%%%%%%%%%%%%%%% END OF PREAMBLE %%%%%%%%%%%%%%%%%%%%%

\begin{document}

%%%%%%%%%%%%%%%%% NAME OF APPLICANT %%%%%%%%%%%%%%%%%%%

\noindent  \LARGE{\textbf{Vasilis Kontonis}}  \\
\vspace{-2ex}
%\hline
\normalsize

%%%%%%%%%%%%%%%%% CONTACT INFORMATION %%%%%%%%%%%%%%%%%

\begin{center}
\begin{tabular}{l l}
 University of Texas at Austin& \hspace{1in} \href{mailto:vkonton@gmail.com}{vkonton@gmail.com} \\
 Department of Computer Sciences & \hspace{1in}  \href{https://vkonton.github.io}{https://vkonton.github.io}\\ 
 2317 Speedway, Austin, TX, USA & \hspace{1in} Cell: +1 (608) 982-4267 \\
 \textbf{Citizenship: Greek} & \hspace{1in} \textbf{Visa: F1/OPT}
\end{tabular}
\end{center}


%%%%%%%%%%%%%%%%% MAIN BODY %%%%%%%%%%%%%%%%%%%%%%%%%%%
% The main body is contained in a tabular environment. To move sections onto the next page, simply end the tabular environment and begin a new tabular environment.

\noindent \begin{longtable}{@{} l l}

\href{https://www.ifml.institute}{\emph{\textbf{University of}}} 
  & \textbf{IFML Postdoctoral Fellow} \\

\href{https://www.ifml.institute}
{\textbf{\emph{Texas Austin}}}  & \emph{2023 -- present, University of Texas at Austin} \\
& Working on the foundations of reliable and efficient ML.
 My research aims at \\ 
& designing algorithms to 
learn from  imperfect or corrupted datasets such
 as learning \\
 & from noisy labels or from biased or censored data. I am particularly interested in \\
 & developing principled, efficient, and practical algorithms. \\
  & \\

\href{https://research.google}{\emph{\textbf{Google}}}
  &\textbf{Research Intern} \\
  & \emph{2022, Google Research, Mountain View}\\
  & Worked on semi-supervised distillation.  Developed a new practical distillation \\ 
  & method called Student Label Mixing Distillation (SLaM) that improved the SOTA \\
  & (published at NeurIPS 2023) and used it to improve distillation in Google products. \\
  & \\

  \href{https://grnet.gr/}{\emph{\textbf{Grnet}}}
  &\textbf{Systems and Services Engineer} \\
  & \emph{2016, Greek Research and Technology Network, Athens}\\
  & Monitoring services, network, and servers, automation of procedures, development \\ 
  & and maintenance of administration tools. \\
  &\\

 \Large{Education}

     & \textbf{Ph.D., Computer Science}, 2018 -- 2023 \\
      \emph{\textbf{University of}}
     & Thesis: Learning From Imperfect Data: 
      Noisy Labels, Truncation, and Coarsening \\
      \emph{\textbf{Wisconsin-Madison} }
     & Advisor: Christos Tzamos\\
     & \\

      \emph{\textbf{National Technical}}
     & \textbf{MEng, Electrical \& Computer Engineering}, 2011 -- 2017\\
     \emph{\textbf{Uni. of Athens} }
     & Thesis: Learning Powers of Poisson Binomial Distributions\\
     & Advisor: Dimitris Fotakis \\
     & \\

% \Large{Research Interests} & \textbf{Machine Learning, Statistics, 
% Theoretical Computer Science}
% & \\

 \Large{Selected}  
 &\textbf{Smoothed Analysis for Learning Concepts with
 Low Intrinsic Dimension} \\
\Large{Publications \footnote{ 
See \href{https://vkonton.github.io}{vkonton.github.io} and
my \href{https://scholar.google.com/citations?user=7_44KWAAAAAJ&hl=el}{Google scholar} 
for the complete list.  
Alphabetical author order unless otherwise.
}}
 & Gautam Chandrasekaran, Adam Klivans, Vasilis Kontonis, \\
  \emph{\textbf{COLT 2024}} 
 & Raghu Meka, Konstantinos Stavropoulos \\
  \emph{\textbf{Best Paper Award}}
 & \emph{37th Annual Conference on Learning Theory} \\
 & \\

  \emph{\textbf{NeurIPS 2023}}
 &\textbf{Optimizing Solution-Samplers for Combinatorial Problems:}\\
  \emph{\textbf{Oral Presentation}} 
 &\textbf{The Landscape of Policy Gradient Methods}\\
 & Constantinos Caramanis, Dimitris Fotakis, Alkis Kalavasis\\
 & Vasilis Kontonis, Christos Tzamos\\
 & \emph{37th Conference on Neural Information Processing Systems} \\
 & \\

 \emph{\textbf{NeurIPS 2023}}
 &\textbf{SLaM: Student Label Mixing for Distillation with Unlabeled Examples} \\
 & Vasilis Kontonis, Fotis Iliopoulos, Khoa Trinh \\
 & Cenk Baykal, Gaurav Menghani, Erik Vee\\ 
 & \emph{37th Conference on Neural Information Processing Systems} \\
 & \\


  \emph{\textbf{NeurIPS 2022}} &\textbf{Linear Label Ranking with Bounded Noise}\\
  \emph{\textbf{Oral Presentation}} 
 &  Dimitris Fotakis, Alvertos Kalavasis, Vasilis Kontonis, Christos Tzamos \\
 & \emph{36th Conference on Neural Information Processing Systems} \\
 & \\

\emph{\textbf{STOC 2022}}
 &\textbf{Learning General Halfspaces with General Massart Noise}\\
 & \textbf{under the Gaussian Distribution}\\
 & Ilias Diakonikolas, Daniel M. Kane, Vasilis Kontonis,
  Christos Tzamos, Nikos Zarifis \\
 & \emph{54th Annual ACM Symposium on Theory of Computing} \\
 & \\


\emph{\textbf{COLT 2021}}
 &\textbf{A Statistical Taylor's Theorem and Extrapolation of Truncated Densities}\\
 & Costis Daskalakis, Vasilis Kontonis, Christos Tzamos, Manolis Zampetakis \\
 & \emph{34th Annual Conference on Learning Theory, } \\
 & \\

\emph{\textbf{COLT 2021}}
 &\textbf{Efficient Algorithms for Learning from Coarse Labels}\\
 &  Dimitris Fotakis, Alvertos Kalavasis, Vasilis Kontonis, Christos Tzamos \\
 & \emph{34th Annual Conference on Learning Theory } \\
 & \\

\emph{\textbf{STOC 2021}}
     &\textbf{Learning Halfspaces with Tsybakov Noise}\\
     & Ilias Diakonikolas, Vasilis Kontonis, Christos Tzamos, Nikos Zarifis \\
     & \emph{53rd Annual ACM Symposium on Theory of Computing } \\
     & \\

\emph{\textbf{NeurIPS 2020}}
     &\textbf{Non-Convex SGD Learns Halfspaces with Adversarial Label Noise} \\
     & Ilias Diakonikolas, Vasilis Kontonis, Christos Tzamos, Nikos Zarifis \\
     & \emph{34th Conference on Neural Information Processing Systems} \\
     & \\

\emph{\textbf{COLT 2020}}
     &\textbf{Learning Halfspaces with Massart Noise}\\
     &\textbf{Under Structured Distributions}\\
     & Ilias Diakonikolas, Vasilis Kontonis, Christos Tzamos, Nikos Zarifis \\
     & \emph{33th Annual Conference on Learning Theory} \\
     & \\

\emph{\textbf{FOCS 2019}}
     & \textbf{Truncated Statitistics with Unknown Truncation} \\
     & Vasilis Kontonis, Christos Tzamos, Manolis Zambetakis\\
     & \emph{60th Annual IEEE Symposium on Foundations of Computer Science} \\
     & \\


  \Large{Teaching}

  & Spring 2022, TA \emph{Introduction to Artificial Intelligence} \\
  \emph{\textbf{University of}}
  & Fall 2021, TA \emph{Introduction to Numerical Methods} \\
  \emph{\textbf{Wisconsin-Madison}}
  & Spring 2021, Spring 2020 Spring 2019, Fall 2018, TA \emph{Introduction to Algorithms}\\
  & Fall 2020, TA \emph{Programming II}\\
   & \\
   \emph{\textbf{National Technical }}
  & Fall 2017, TA  \emph{Algorithms and Complexity}  \\
  \emph{\textbf{Uni. of Athens} }
  & Fall 2016, TA \emph{Operating Systems}  \\
  & \\


\Large{Awards}

 & Best Paper Award at Conference on Learning Theory 2024 (Edmonton, CA) \\

 & Recipient of the Bodossakis Fellowship 2022 \\

 & Recipient of the Gerondellis Fellowship 2020 \\

 & Recipient of the Eurobank Grant ``The great moment for Education" in 2010\\
 & for graduating first in my high-school. \\

 & Recipient of the Touramanoglou Grant in 2010 for ranking among the top \\
 & high school graduates of the cities of Ilioupolis and Ymittos. \\
 &\\

  \Large{Languages}   & English, Greek (native), German \\
\Large{and Skills}    & Python, Pytorch, Tensorflow, \LaTeX, GNU/Linux\\

\end{longtable}


\end{document}
